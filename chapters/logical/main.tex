\section{Бинаризация признаков}

Бинаризация признаков – это процесс преобразования исходных признаков в бинарные переменные, которые принимают значения \(0\) или \(1\). 
%Этот метод широко используется в задачах %машинного обучения, особенно в логических %методах классификации, где входные данные %должны быть представлены в виде набора %булевых предикатов.

\subsection{Бинаризация количественных признаков}

Для признака \( f: X \to D_f \), где \( D_f \) – множество возможных значений признака, бинаризация заключается в создании предикатов, проверяющих выполнение определённых условий. Эти предикаты позволяют разбить множество значений признака на подмножества, которые можно использовать в логических моделях.

В зависимости от типа признака, бинаризация осуществляется следующим образом:
\begin{itemize}
    \item \textbf{Номинальный признак} (\(f\) принимает конечное множество значений, без упорядоченности):
    \[
    \beta(x) = [f(x) = d], \quad d \in D_f;
    \]
    \[
    \beta(x) = [f(x) \in D'], \quad D' \subset D_f.
    \]
    \item \textbf{Порядковый или количественный признак} (\(f\) принимает значения, между которыми можно определить порядок):
    \[
    \beta(x) = [f(x) \leq d], \quad d \in D_f;
    \]
    \[
    \beta(x) = [d \leq f(x) \leq d'], \quad d, d' \in D_f, \, d < d'.
    \]
\end{itemize}

Для количественных признаков (\(f: X \to \mathbb{R}\)) важно выбирать такие пороговые значения \(d\), которые разделяют выборку на значимые группы. Например, 
%пороги \(d\) могут быть определены как средние значения между %соседними элементами вариационного ряда \(f(x_1), \dots, %f(x_\ell)\), упорядоченного по возрастанию:
\[
d_i = \frac{f^{(i)} + f^{(i+1)}}{2}, \quad f^{(i)} \neq f^{(i+1)}, \; i = 1, \dots, \ell - 1,
\]
где \(f^{(1)} \leq f^{(2)} \leq \dots \leq f^{(\ell)}\) – упорядоченные значения признака. (См. рис)

Такими способами можно получить много разных предикатов. Мы хотим выбрать из них самые ''лучшие'' (в каком-либо смысле). Для этого разобьем диапазон значений признака на зоны.

\begin{figure}
    \centering
    \includegraphics[scale = 1]{images/bin1.png}
    \caption{Вариационный ряд значений признака $f(x)$ и пороги $d_i$}
\end{figure}

\subsection{Разбиение диапазона значений признака на зоны}

Каждая зона определяется бинарным предикатом:
\begin{align*}
\zeta_0(x) &= [f(x) < d_1], \\
\zeta_s(x) &= [d_s \leq f(x) < d_{s+1}], \quad s = 1, \dots, r-1, \\
\zeta_r(x) &= [d_r \leq f(x)].
\end{align*}

Способы разбиения:
\begin{itemize}
    \item Жадная максимизация информативности путем слияний
    \item Разбиение на равномощные подвыборки
    \item Разбиение по равномерной сетке ''удобных'' значений (например, с минимальным числом значащих цифр)
    \item Объединение нескольких разбиений
\end{itemize}

\subsection{Жадный алгоритм слияния зон}

Алгоритм начинает с разбиения на ''мелкие'' зоны. Пороги проходят между всеми соседними парами точек, принадлежащих \emph{разным} классам, т.~к. расстановка порогов между точками одного класса приведет только к уменьшению информативности зон. Далее зоны укрупняются путём слияния \emph{троек} соседних зон. Зоны сливаются до тех пор, пока
информативность некоторой слитой зоны превышает информативность
исходных зон, либо пока не будет получено заданное количество зон $r$. Каждый раз сливается тройка, дающая наибольший выигрыш в информативности.

\begin{figure}
    \centering
    \includegraphics[scale = 1]{images/bin2.png}
    \caption{Начальное разбиение на зоны}
\end{figure}

\newpage
\textbf{Вход:}
\begin{itemize}
    \item $f(x)$ — признак;
    \item $c \in Y$ — выделенный класс;
    \item $X^\ell = \{(x_i, y_i)\}_{i=1}^\ell$ — выборка, упорядоченная по возрастанию $f(x_i)$;
    \item $r$ — желаемое количество зон;
    \item $\delta_0$ — порог слияния зон (по умолчанию $\delta_0 = 0$).
\end{itemize}

\textbf{Выход:}
\[
D = \{d_1, \dots, d_n\} \text{ — строго возрастающая последовательность порогов;}
\]

\hline

\begin{enumerate}
    \item $D := \emptyset;$
    \item \textbf{для всех} $i = 2, \dots, \ell$:
    \begin{itemize}
        \item \textbf{если} $f(x_{i-1}) \neq f(x_i)$ и $[y_{i-1} = c] \neq [y_i = c]$ \textbf{то}
        \begin{itemize}
            \item добавить новый порог $d := \frac{f(x_{i-1}) + f(x_i)}{2}$ в конец последовательности $D$;
        \end{itemize}
    \end{itemize}
    \item \textbf{повторять}
    \begin{enumerate}
        \item \textbf{для всех} $d_i \in D, i = 1, \dots, |D| - 1$:
        \begin{itemize}
            \item вычислить выигрыш от слияния тройки соседних зон $\zeta_{i-1}, \zeta_i, \zeta_{i+1}$:
            \[
            \delta_i := I_c(\zeta_{i-1} \cup \zeta_i \cup \zeta_{i+1}) - \max\{I_c(\zeta_{i-1}), I_c(\zeta_i), I_c(\zeta_{i+1})\};
            \]
        \end{itemize}
        \item найти тройку зон, для которой слияние наиболее выгодно:
        \[
        i := \arg \max \delta_i;
        \]
        \item \textbf{если} $\delta_i > \delta_0$ \textbf{то}
        \begin{itemize}
            \item слить зоны $\zeta_{i-1}, \zeta_i, \zeta_{i+1}$, удалить пороги $d_i$ и $d_{i+1}$ из последовательности $D$;
        \end{itemize}
    \end{enumerate}
    \item \textbf{пока} $|D| > r + 1$.
\end{enumerate}

\subsection{Задачи}

\textbf{Задача 1}
Предположим, вы владелец интернет-магазина, и у вас есть данные о стоимости товаров и их популярности (популярен — \( 1 \), непопулярен — \( 0 \)). 
Для анализа спроса вы хотите разбить товары на ценовые зоны, чтобы лучше понять поведение покупателей.

Данные представлены в таблице:

\[
\begin{array}{|c|c|c|}
\hline
\text{№ товара} & \text{Цена товара (\$)} & \text{Популярность } y \\
\hline
1 & 10 & 1 \\
2 & 12 & 1 \\
3 & 15 & 0 \\
4 & 17 & 1 \\
5 & 20 & 0 \\
6 & 23 & 0 \\
7 & 25 & 1 \\
\hline
\end{array}
\]

Как алгоритм слияния зон первично разобьёт выборку на ценовые зоны?

\textbf{Решение}
Рассчитаем пороги:

\[
\begin{aligned}
&d_1 = \frac{12 + 15}{2} = 13.5, \\
&d_2 = \frac{15 + 17}{2} = 16.0, \\
&d_3 = \frac{17 + 20}{2} = 18.5, \\
&d_4 = \frac{23 + 25}{2} = 24.0
\end{aligned}
\]

На основе рассчитанных порогов получаем зоны:

\[
\begin{aligned}
&\zeta_0(x) = [\text{Цена} < 13.5], \\
&\zeta_1(x) = [13.5 \leq \text{Цена} < 16.0], \\
&\zeta_2(x) = [16.0 \leq \text{Цена} < 18.5], \\
&\zeta_4(x) = [29.0 \leq \text{Цена} < 24.0], \\
&\zeta_5(x) = [\text{Цена} \geq 24.0].
\end{aligned}
\]

\textbf{Задача 2}
Будут ли разбиения диапазона меняться в зависимости от класса, относительно которого они производятся? Как изменить алгоритм для получения ''универсального'' разбиения, учитывающего сразу все классы? 

\textbf{Решение}
Да, будут, т.~к. информативность зависит от класса. Нужно заменить критерий информативности многоклассовым критерием.

\textbf{Задача 3}
Какую сложность имеет алгоритм слияния зон? Как можно его ускорить?

\textbf{Решение}
Этот алгоритм имеет трудоёмкость $O(l^2)$. Его можно заметно ускорить, если на каждой итерации сливать не одну тройку зон, а $\tau l$ троек с достаточно большим выигрышем $\delta I_i$, при условии, что они не перекрываются. В этом случае трудоёмкость составляет $O(l / \sqrt{\tau})$.

\section{Взвешенное голосование правил}

Допустим, имеется консилиум экспертов, каждый член которого может допустить ошибку. Процедура голосования — это способ повышения качества принимаемых решений, при котором ошибки отдельных экспертов компенсируют друг друга.

Ранее принцип голосования применялся для построения композиций из произвольных алгоритмов классификации. Теперь рассмотрим композиции, состоящие из логических закономерностей.

\subsection{Принцип голосования}

Пусть для каждого класса $c \in Y$ построено множество логических закономерностей (правил), специализирующихся на различении объектов данного класса:

\[
R_c = \{ \varphi_{tc} : X \to \{0, 1\} \mid t = 1, \dots, T_c \}
\]

Считается, что если $\varphi_{tc}(x) = 1$, то правило $\varphi_{tc}$ относит объект $x \in X$ к классу $c$. Если же $\varphi_{tc}(x) = 0$, то правило воздерживается от классификации объекта $x$.

Алгоритм простого голосования (simple voting) подсчитывает долю правил в наборах $R_c$, относящих объект $x$ к каждому из классов:

\[
\Gamma_c(x) = \frac{1}{T_c} \sum_{t=1}^{T_c} \varphi_{tc}(x), \quad c \in Y,
\]

и относит объект $x$ к тому классу, за который подана наибольшая доля голосов:

\[
a(x) = \arg \max_{c \in Y} \Gamma_c(x).
\]

Если максимум достигается одновременно на нескольких классах, выбирается тот, для которого цена ошибки меньше.

Нормирующий множитель $\frac{1}{T_c}$ вводится для того, чтобы наборы с большим числом правил не перетягивали объекты в свой класс.

\subsection{Алгоритм взвешенного голосования}
Алгоритм взвешенного голосования (weighted voting, WV) действует более тонко, учитывая, что правила могут иметь различную ценность. Каждому правилу $\varphi_{tc}$ приписывается вес $\alpha_{tc} \geq 0$, и при голосовании берется взвешенная сумма голосов:

\[
\Gamma_c(x) = \sum_{t=1}^{T_c} \alpha_{tc} \varphi_{tc}(x), \quad \alpha_{tc} > 0.
\]

Веса нормируются на единицу:

\[
\sum_{t=1}^{T_c} \alpha_{tc} = 1, \quad \forall c \in Y.
\]

Поэтому функцию $\Gamma_c(x)$ называют также выпуклой комбинацией правил $\varphi_1, \dots, \varphi_{T_c}$. Очевидно, простое голосование является частным случаем взвешенного, когда веса одинаковы и равны $\frac{1}{T_c}$.

На первый взгляд, вес правила должен определяться его информативностью. Однако, важно также учитывать, насколько данное правило уникально. Если имеется 10 хороших, но одинаковых (или почти одинаковых) правил, их суммарный вес должен быть сравним с весом столь же хорошего правила, не похожего на все остальные. Таким образом, веса должны учитывать не только ценность правил, но и их различность.

Простой общий подход к настройке весов заключается в том, чтобы сначала найти набор правил $\{ \varphi_{tc}(x) \}$, затем принять их за новые (бинарные) признаки и построить в этом новом признаковом пространстве линейную разделяющую поверхность (кусочно-линейную, если $|Y| > 2$). Для этого можно использовать логистическую регрессию, однослойный персептрон или метод опорных векторов. Существуют и другие подходы. Например, в разделе 1.5.4 будет рассмотрен метод бустинга, в котором правила настраиваются последовательно, и для каждого правила сразу вычисляется его вес.

\subsection{Проблема диверсификации правил}
Голосующие правила должны быть существенно различны, иначе они будут бесполезны для классификации. Продолжая аналогию с консилиумом, заметим, что нет никакого смысла держать в консилиуме эксперта A, если он регулярно подсматривает решения у эксперта B.

Приведем простое теоретико-вероятностное обоснование принципа диверсификации, или повышения различности (diversity) правил \cite{14}. Пусть $X$ — вероятностное пространство, множество ответов $Y$ конечно. Введем случайную величину $M(x)$, равную перевесу голосов в пользу правильного класса; её называют также отступом (margin) объекта $x$ от границы классов:

\[
M(x) = \Gamma_c(x) - \Gamma_{\overline{c}}(x), \quad \Gamma_{\overline{c}}(x) = \max_{y \in Y \setminus \{c\}} \Gamma_y(x), \quad c = y^*(x).
\]

Если отступ положителен ($M(x) > 0$), то алгоритм голосования правильно классифицирует объект $x$. Предположим, что в среднем наш алгоритм классифицирует хотя бы немного лучше, чем наугад: $E[M] > 0$. Тогда можно оценить вероятность ошибки по неравенству Чебышева:

\[
P\{M < 0\} \leq P\{|E[M] - M| > E[M]\} \leq \frac{D_M}{(E[M])^2}.
\]

Отсюда вывод: для уменьшения вероятности ошибки необходимо максимизировать ожидание перевеса голосов $E[M]$ и минимизировать его дисперсию $D_M$. Для выполнения этих условий каждый объект должен выделяться примерно одинаковым числом правил. Обычно ни одно из правил не выделяет класс целиком, поэтому правила должны быть существенно различны, то есть выделять существенно различные подмножества объектов.

Неплохая эвристика, усиливающая различия между правилами и позволяющая равномернее выделять объекты обучения, используется в алгоритме CORAL \cite{12}. Сначала для фиксированного класса $c \in Y$ строится покрывающий набор правил точно так, как это делалось для решающих списков. Затем строится второй покрывающий набор, но при этом запрещается использовать признаки, часто входившие в закономерности первого набора. Поэтому второй набор неминуемо окажется отличным от первого. Затем запрещаются признаки, часто входившие в оба набора, и строится третий набор. И так далее, для каждого класса $c \in Y$.

\subsection{Отказы от классификации}
Возможны ситуации, когда ни одно из правил не выделяет классифицируемый объект $x$. Тогда алгоритм должен либо отказываться от классификации, либо относить объект к классу, имеющему наименьшую цену ошибки. Отказ алгоритма означает, что данный объект является нетипичным, не подпадающим ни под одну из ранее обнаруженных закономерностей. Вообще, обнаружение нетипичности (novelty detection) принято считать отдельным видом задач обучения по прецедентам, наряду с классификацией и кластеризацией. Способность алгоритмов отказываться от классификации нетипичных объектов во многих приложениях является скорее преимуществом, чем недостатком. В то же время, число отказов не должно быть слишком большим.

Итак, при построении алгоритмов взвешенного голосования правил возникает четыре основных вопроса:
\begin{itemize}
    \item Как построить много правил по одной и той же выборке?
    \item Как избежать повторов и построения почти одинаковых правил?
    \item Как избежать появления непокрытых объектов и обеспечить равномерное покрытие всей выборки правилами?
    \item Как определять веса правил при взвешенном голосовании?
\end{itemize}

Рассмотрим, как эти проблемы решаются в известных алгоритмах в следующих параграфах.

\subsection{Задачи}

\textbf{Задача 1: Алгоритм простого голосования}

Допустим, для класса $c \in Y$ существуют 10 правил, которые используют данные для классификации. Каждое правило возвращает 1 или 0 для объекта $x$. Если для объекта $x$ правила 3 и 7 верно классифицируют объект как класс $c$, а остальные возвращают 0, как будет рассчитана доля голосов для класса $c$?

\textbf{Решение:}  
Для класса $c$ доля голосов рассчитывается как сумма всех правил, которые классифицируют объект как класс $c$, делённая на общее количество правил. Если 10 правил, то:
\[
\Gamma_c(x) = \frac{1}{10} \left( 1 + 0 + 0 + 0 + 0 + 0 + 1 + 0 + 0 + 0 \right) = \frac{2}{10} = 0.2
\]
Таким образом, для объекта $x$ доля голосов для класса $c$ составит 0.2.

\textbf{Задача 2: Проблема с весами в алгоритме взвешенного голосования}

В алгоритме взвешенного голосования веса для каждого правила нормируются на единицу. Если для класса $c$ у нас есть 3 правила с весами $\alpha_1 = 0.5$, $\alpha_2 = 0.3$, и $\alpha_3 = 0.2$, как будет выглядеть итоговая сумма голосов $\Gamma_c(x)$, если объект $x$ классифицируется всеми тремя правилами как класс $c$?

\textbf{Решение:}  
Итоговая сумма голосов для класса $c$ рассчитывается по формуле:
\[
\Gamma_c(x) = \sum_{t=1}^{T_c} \alpha_{tc} \varphi_{tc}(x)
\]
где $\varphi_{tc}(x) = 1$, если правило классифицирует объект как класс $c$, и 0 в противном случае. Если все 3 правила классифицируют объект как $c$, то:
\[
\Gamma_c(x) = 0.5 + 0.3 + 0.2 = 1.0
\]

\textbf{Задача 3: Проблема диверсификации правил}

Какова вероятность ошибки при использовании алгоритма голосования, если все правила сильно похожи друг на друга (например, классифицируют одинаковые подмножества объектов)?

\textbf{Решение:}  
Если правила сильно похожи, то вероятность ошибки возрастает. В таких случаях, возможно, правило не будет существенно различать объекты, и алгоритм может ошибаться при классификации новых объектов. Чтобы уменьшить вероятность ошибки, правила должны быть разнообразными, то есть они должны выделять разные подмножества объектов. Для максимизации различий между правилами можно использовать метод, как в алгоритме CORAL, который строит покрывающие наборы правил, постепенно исключая часто встречающиеся признаки.

\textbf{Задача 4: Принцип диверсификации правил}

Пусть для объекта $x$ имеется 10 правил, из которых 8 классифицируют объект как класс $c$, а остальные 2 — как класс $d$. Какой отступ $M(x)$ будет при расчете вероятности правильной классификации?

\textbf{Решение:}  
Отступ для объекта $x$ определяется как разница между голосами для правильного класса и максимальным голосом для всех остальных классов:
\[
M(x) = \Gamma_c(x) - \Gamma_{\overline{c}}(x)
\]
Если из 10 правил 8 голосуют за класс $c$ (доля голосов $0.8$) и 2 — за класс $d$ (доля голосов $0.2$), то:
\[
M(x) = 0.8 - 0.2 = 0.6
\]
Если отступ положителен ($M(x) > 0$), то классификация будет правильной.

\textbf{Задача 5: Отказы от классификации}

Если для объекта $x$ не существует правила, которое его классифицирует, что должен делать алгоритм голосования? Как можно обработать такой случай?

\textbf{Решение:}  
В таком случае алгоритм может либо отказаться от классификации, либо отнести объект к классу с наименьшей ценой ошибки. Такой отказ от классификации часто называют обнаружением нетипичности (novelty detection), что является отдельной задачей в машинном обучении. Важно, чтобы количество отказов не было слишком большим, так как это может снизить эффективность алгоритма.


\section{Решающие списки}
\subsection{Опреление}
    Решающий список - это логический алгоритм классификации $a: X \xrightarrow{} Y$, задаваемый закономерностями $\varphi_1, \dots, \varphi_T$ классов $c_1, \dots, c_T \in Y$, вычисляемый следующим алоритмом

\hline
\begin{enumerate}
    \item \textbf{для всех} $t = 1, \dots, T_{max}$:
    \begin{itemize}
        \item \textbf{если} $\varphi_t(x) = 1$ 
        \begin{itemize}
            \item \textbf{вернуть $c_t$}
        \end{itemize}
    \end{itemize}
    
    \item \textbf{вернуть $c_0$}
\end{enumerate}
\hline
«Особый ответ» $c_0$ означает отказ алгоритма от классификации объекта $x$.
Обычно такие объекты приписывают классу, имеющему минимальную цену ошибки.
Например, в задаче выдачи кредитов отказ алгоритма приведёт к более осторожному решению «не выдавать». В задаче распознавания спама более осторожным будет
решение «не спам».

\textbf{Замечание} Соседние правила в списке $\varphi_{t-1}, \varphi_t$ можно переставлять местами,
если только они приписаны к одному классу, $c_{t-1} = c_t$. В общем случае перестановка
правил в списке изменяет алгоритм.

\subsection{Жадный алгоритм посторения}
Рассмотрим жадный алгоритм построения решающих списков.

Алгоритм, приведённый ниже, на каждой итерации строит ровно одно правило $\varphi_t$, выделяющее максимальное число объектов некоторого класса $c_t$ и минимальное число объектов
всех остальных классов. Для этого на каждой итерации производится поиск наиболее информативного правила $\varphi_t \in \Phi$, допускающего относительно мало ошибок. Семейство
правил $\Phi$ может быть каким угодно, лишь бы для него существовала эффективная
процедура поиска закономерностей. После построения правила $\varphi_t$ выделенные им объекты изымаются из выборки и алгоритм переходит к поиску следующего правила $\varphi_{t+1}$ по оставшимся объектам. В итоге выборка оказывается покрытой
множествами вида $\{x: \varphi_t(x) = 1\}$. Поэтому решающий список называют также покрывающим набором закономерностей или машиной покрывающих множеств.

\hline
\textbf{Вход:}
\begin{itemize}
    \item $X^l$ -- выборка;
    \item $T_{max}$ -- максимальное допустимое число правил в списке;
    \item $I_{min}$ -- минимальная допустимая информативность правил в списке;
    \item $E_{max}$ -- максимальная допустимая доля ошибок на обучающей выборке;
    \item $l_0$ -- максимальное допустимое число отказов.
\end{itemize}

\textbf{Выход:}
 $\{\varphi_1, \dots, \varphi_T\} \text{ — искомые решающие правила;}$

\hline
\begin{enumerate}
    \item $U := X^l;$
    \item \textbf{для всех} $i = 1, \dots, T_{max}$:
    \begin{itemize}
        \item \text{вычираем класс} $c := c_t$
        \item найти наиболее информативное правило при ограничении на долю ошибок:\
        $\varphi_t := \arg\min_{\varphi \in \Phi'} I_c(\varphi, U)$, где $\Phi' = \{ \varphi \in \Phi \ | \ E_c(\varphi, U) \leq E_{max} \}$ 

        \item \textbf{если} $I_c(\varphi_t, U) \leq I_{min}$ \textbf{то выход}
        \item Исключить из выборки объекты, выделенные правилом $\varphi_t$:
        $U := \{x \in U | \varphi_t(x) = 0\}$

        \item \textbf{если} $|U| \leq l_0$ \textbf{то выход}
    \end{itemize}
\end{enumerate}
\hline

\subsection{Анализ алгоритма}

\textbf{Критерии отбора правил.} Почему приходится использовать сразу два критерия
отбора правил $I_c$ и $E_c$? Правило с высокой информативностью $I_c$ вполне может допускать значительную долю ошибок $E_c$. Это нежелательно, так как в решающем списке каждое правило принимает окончательное решение. С другой стороны, правило с небольшой долей ошибок может выделять слишком мало объектов,
и по этой причине не являться закономерностью. Совместное использование обоих
критериев позволяет отобрать предикаты, удовлетворяющие условиям как статистической, так и логической закономерности.

\textbf{Критерии останова.} В данном алгоритме одновременно работают три критерия останова: 

(1) построение заданного числа правил $T_{max}$; 

(2) покрытие всей выборки, за исключением не более $l_0$ объектов; 

(3) невозможность найти правило с информативностью выше Imin по остатку выборки.

\textbf{Оптимизация сложности решающего списка.} Параметр $E_{max}$ позволяет найти
компромисс между точностью классификации обучающего материала и сложностью
списка. Уменьшение $E_{max}$ приводит к снижению числа ошибок на обучении. С другой
стороны, оно ужесточает отбор правил, способствует уменьшению числа объектов,
выделяемых отдельными правилами, и увеличению длины списка $T$. Правила, выделяющие слишком мало объектов, статистически не надёжны и могут допускать много
ошибок на независимых контрольных данных. Иными словами, увеличение длины
списка при одновременном «измельчении» правил может приводить к эффекту переобучения. Из общих соображений Emax должно быть приблизительно равно доле
ошибок, которую мы ожидаем получить как на обучающей выборке, так и вне её.
На практике параметр Emax подбирается экспериментально.

\textbf{Стратегия выбора класса.} Мы ничего не сказали о том, как выбирается класс $c_t$. Рассмотрим два варианта.
Первый вариант — сначала строятся все правила для первого класса, затем для второго, и так далее. Классы берутся в порядке убывания важности или цены ошибки. Преимущество данного варианта в том, что правила оказываются независимыми — в пределах своего класса их можно переставлять местами. Это улучшает
интерпретируемость правил.
Второй вариант — совместить 2 шага и выбирать пару $(\varphi_t
, c_t) \in \Phi \times Y$ , для
которой информативность $I_{c_t}(\varphi_t, U)$ максимальна. Тогда правила различных классов
могут следовать вперемежку. Доказано, что списки такого типа реализуют более широкое множество функций. При этом улучшается разделяющая способность
списка, но ухудшается его интерпретируемость.
На практике первый вариант часто оказывается более удобным. В некоторых случаях правила строятся только для $(M − 1)$ классов, а в последний, наименее
важный, класс $c_0$ объекты заносятся «по остаточному принципу».
Обработка пропусков в данных. Решающие списки позволяют легко обойти проблему пропущенных данных. Если для вычисления предиката 
$\varphi_t(x)$ не хватает данных, то считается, что $\varphi_t(x) = 0$, и обработку объекта $x$ берут на себя следующие
правила в списке. Это относится и к стадии обучения, и к стадии классификации.

\subsection{Разновидности решающих списков}
Логику решающего списка или, что то же самое, комитета старшинства, имеют
многие алгоритмы, предлагавшиеся в разное время разными авторами под разными названиями. Многочисленные варианты отличаются выбором семейства предикатов $\Phi$, критерием информативности и методом поиска информативных предикатов.

\textbf{Пример} Наиболее распространены решающие списки конъюнкций. Они почти
идеально соответствуют человеческой логике принятия решений, основанной на последовательной проверке достаточно простых правил. Поэтому решающие списки
часто используются для представления знаний, извлекаемых непосредственно из эмпирических данных. Для построения отдельных правил можно использовать жадный алгоритм, применяя для поиска $\varphi$ любой из методов поиска информативных конъюнкций,
например.

\textbf{Пример} В алгоритме Маршанда перебираются всевозможные гиперплоскости, разделяющие какие-нибудь три точки (data dependent half-spaces), и из них выбирается полуплоскость с максимальной информативностью.

\subsection{Достоинства и недостатки}

\textbf{Достоинства решающих списков.}
\begin{enumerate}
    \item Интерпретируемость и простота классификации. Обученное по выборке правило классификации можно записать в виде инструкции и выполнять «вручную».
    \item Гибкость: в зависимости от выбора множества $\Phi$ можно строить весьма разнообразные алгоритмические конструкции.
    \item Возможность обработки разнотипных данных и данных с пропусками.
\end{enumerate}
\\
\textbf{Недостатки решающих списков.}
\begin{enumerate}
    \item Если множество правил $\Phi$ выбрано неудачно, список может не построиться. При этом возможен высокий процент отказов от классификации.
    \item Возможна утрата интерпретируемости, если список длинный и правила различных классов следуют вперемежку. В этом случае правила не могут быть
    интерпретированы по-отдельности, без учёта предшествующих правил, и логика их взаимодействия становится довольно запутанной.
    \item Каждый объект классифицируется только одним правилом, что не позволяет
    правилам компенсировать неточности друг друга. Данный недостаток устраняется путём голосования правил, но это уже совсем другой алгоритм.
\end{enumerate}

\subsection{Задачи}

\textbf{Задача 1}
Постройте решающий список для логической функции $x \vee y \vee \overline{z}.$

\textbf{Решение}

\includegraphics[scale = 0.5]{images/decide_list_task1_sol.png}

\textbf{Задача 2}
Адаптируйте решающий список для задач регрессии.

\textbf{Решение}
Давайте разобъём множество значений искомой зависимости $y: X \rightarrow{} Y$ на $T + 1$ множеств вида:
$[y < d_1], [d_1 \leq y < d_2], ..., [d_{T-1} \leq y < d_T], [d_T \leq y].$ 
Сопоставим эти множества с классами $c_1, \dots, c_T$.

Теперь запустим алгоритм построения решающего списка для полученных классов, 
но помимо нахождения правил $\varphi$, будем так же находить среднее значение $y$ на объектах, которые удовлетворяют этому правилу.
И будем выдавать по объекту $x$ среднее значение для объектов этого класса.

\section {Решающие таблицы}
\subsection{Опреление}

Решающая таблица - это частный стлучай решающего дерева глубины $H$, для всех узлов уровня $h$ условия ветвления $f_h(x)$ одинаковы. На уровне $h$ ровно $2^{h-1}$ вернин. $X$ делится на $2^H$ "ячеек".

\textbf{Пример.} Задача XOR, $H = 2$.

    \includegraphics[scale = 0.5]{images/decide_table_exapmle.png}

\subsection{Жадный алгоритм посторения}
Рассмотрим жадный алгоритм построения решающей таблицы

\hline
\textbf{Вход:}
\begin{itemize}
    \item $X^l$ — выборка;
    \item $F$ - множество признаков ;
    \item $H$ - глубина дерева.
\end{itemize}

\textbf{Выход:}
\begin{itemize}
    \item $\{f_1, \dots, f_H\};$ 
    \item таблица $T: \{0, 1\}^H \xrightarrow{} Y$
\end{itemize}

\hline
\begin{enumerate}
    \item $U := X^l;$
    \item \textbf{для всех} $h = 1, \dots, H$:
    \begin{itemize}
        \item предикат с максимальным выигрышем определённости:
        $f_h := \arg\min_{f \in F} Gain(f_1, \dots, f_{h-1}, f)$
    \end{itemize}
    \item классификация по мажоритарному правилу 
    $ T(\beta) := Major(U_{H\beta}) $
\end{enumerate}
\hline
Выигрыш от ветвления на уровне $h$ по всей выборке $X^L$:
$$Gain(f_1, \dots, f_h) = \Phi(X^l) - \sum_{\beta\in \{0, 1\}^h} \frac{U_{h\beta}}{l}\Phi(U_{h\beta})$$

$U_{h\beta} = \{ x_i \in X^l \ | \ f_s(x_i)=\beta_s,\ s=1\dots h \}, \beta=(\beta_1, \dots, \beta_h) \in \{0, 1\}^h$

$p_y =\frac{1}{|U|} \sum_{x_i \ in U}[y_i = y]$ -- частичная оценка $P(y|U)$.

$\Phi(U)$ -- мера неопределённости распределения $p_y$:
\begin{enumerate}
    \item минимальная  равна нулю, когда $p_y \in \{0,1\}$
    \item ксимальна, когда $p_y = \frac{1}{|Y|}$ -- номерное распределение
    \item симметрична, то есть не зависи от перенумерации классов
\end{enumerate}

\subsection{Задачи}

\textbf{Задача}
Постройте решающую таблицу для логической функции $x \vee (y \wedge \overline{z}).$ Изобразите её ввиде дерева.

\textbf{Решение}
\begin{figure}
    \centering
    \includegraphics[width=0.5\linewidth]{decide_table_task_sol.png}
\end{figure}

\section{Алгоритм ТЭМП}

Полный перебор всех конъюнкций ранга не более $K$ требует экспоненциального по $K$ числа операций. В реальных задачах объём вычислений становится огромным уже при $K > 3$, и от идеи полного перебора приходится отказаться.

Существует две стандартные стратегии перебора конъюнкций: поиск в глубину (depth-first search) и поиск в ширину (breadth-first search). Первая применяется в алгоритме КОРА, вторая — в алгоритме ТЭМП, предложенным Г. С. Лбовым в 1976 году. Поиск в ширину работает немного быстрее, и в него легче встраивать различные эвристики, сокращающие перебор.

В исходном варианте алгоритм ТЭМП выполнял полный перебор всех конъюнкций ранга не более $K$. Ниже описан слегка модифицированный вариант, позволяющий ограничить перебор и увеличить максимальный ранг конъюнкций $K$.

Алгоритм 1.9 начинает процесс поиска закономерностей с построения конъюнкций ранга 1. Для этого отбираются не более $T_1$ самых информативных предикатов из базового множества $\mathcal{B}$. Затем к каждому из отобранных предикатов добавляется по одному терму из $\mathcal{B}$ всеми возможными способами. Получается не более $T_1|\mathcal{B}|$ конъюнкций ранга 2, из которых снова отбираются $T_1$ самых информативных. И так далее. На каждом шаге процесса делается попытка добавить один терм к каждой из имеющихся конъюнкций. Наращивание конъюнкций прекращается либо при достижении максимального ранга $K$, либо когда ни одну из конъюнкций не удаётся улучшить путём добавления терма.

Лучшие конъюнкции, собранные со всех шагов, заносятся в списки $R_c$. Таким образом, списки $R_c$ могут содержать конъюнкции различного ранга.

Параметр $T_1$ позволяет найти компромисс между качеством и скоростью работы алгоритма. При $T_1 = 1$ алгоритм ТЭМП работает исключительно быстро и строит единственную конъюнкцию, добавляя термы по очереди. Фактически, он совпадает с жадным Алгоритмом 1.2. При увеличении $T_1$ пространство поиска расширяется, алгоритм начинает работать медленнее, но находит больше информативных конъюнкций. На практике выбирают максимальное значение параметра $T_1$, при котором поиск занимает приемлемое время. Однако стратегия поиска всё равно остаётся жадной — термы оптимизируются по-отдельности, и при подборе каждого терма учитываются только предыдущие, но не последующие термы.

Для улучшения конъюнкций к ним применяют эвристические методы «финальной шлифовки» — стабилизацию и редукцию.

В результате стабилизации конъюнкции становятся локально неулучшаемыми. Алгоритм в целом становится более устойчивым — при незначительных изменениях в составе обучающей выборки он чаще находит одни и те же закономерности, а значит, улучшается его способность обобщать эмпирические факты.

В результате стабилизации некоторые конъюнкции могут совпасть, и в списке появятся дубликаты. Их удаление предусмотрено на шаге 13. Если список $R_c$ поддерживается отсортированным по информативности, то удаление дубликатов является недорогой операцией, так как достаточно проверять на совпадение только соседние конъюнкции с одинаковой информативностью.

Если задать $T_1 = \infty$, то алгоритм выполнит полный перебор, как в исходном варианте ТЭМП. «Финальная шлифовка» в этом случае не нужна.

\subsection{Алгоритм 1.9. Построение списка информативных конъюнкций методом поиска в ширину (алгоритм ТЭМП)}


\subsection{Алгоритм 1.9: Построение списка информативных конъюнкций методом поиска в ширину}

\textbf{Вход:}
\begin{itemize}
    \item $X_\ell$ --- обучающая выборка;
    \item $\mathcal{B}$ --- семейство элементарных предикатов;
    \item $c \in Y$ --- класс, для которого строится список конъюнкций;
    \item $K$ --- максимальный ранг конъюнкций;
    \item $T_1$ --- число лучших конъюнкций, отбираемых на каждом шаге;
    \item $T_0$ --- число лучших конъюнкций, отбираемых на последнем шаге, $T_0 \leq T_1$;
    \item $I_{\min}$ --- порог информативности;
    \item $E_{\max}$ --- порог допустимой доли ошибок;
    \item $X_k$ --- контрольная выборка для проведения редукции.
\end{itemize}

\textbf{Выход:} Список конъюнкций $R_c = \{\varphi_t^c(x) : t = 1, \dots, T_c\}$.

\begin{enumerate}
    \item $R_c \gets \emptyset$;
    \item Для всех $\beta \in \mathcal{B}$: \texttt{Добавить\_в\_список($R_c$, $\beta$, $T_1$)};
    \item Для всех $k = 2, \dots, K$:
    \begin{enumerate}
        \item Для всех конъюнкций $\varphi \in R_c$ ранга $(k-1)$:
        \begin{enumerate}
            \item Для всех предикатов $\beta \in \mathcal{B}$, которых ещё нет в конъюнкции $\varphi$:
            \begin{enumerate}
                \item Добавить терм $\beta$ к конъюнкции $\varphi$: $\varphi' = \varphi \land \beta$;
                \item Если $I_c(\varphi') > I_{\min}$, $E_c(\varphi') \leq E_{\max}$ и конъюнкции $\varphi'$ нет в $R_c$, то \texttt{Добавить\_в\_список($R_c$, $\varphi'$, $T_1$)}.
            \end{enumerate}
        \end{enumerate}
    \end{enumerate}
    \item Для всех конъюнкций $\varphi \in R_c$:
    \begin{enumerate}
        \item Стабилизация($\varphi$);
        \item Редукция($\varphi$, $X^k$);
    \end{enumerate}
    \item Удалить из списка $R_c$ дублирующие конъюнкции;
    \item Оставить в списке $R_c$ не более $T_0$ лучших конъюнкций.
\end{enumerate}


\subsection{Достоинства алгоритма ТЭМП}

\begin{itemize}
    \item ТЭМП существенно более эффективен, чем КОРА, особенно при поиске конъюнкций ранга больше 3. Он решает поставленную задачу за $O(KT_1|B|\\ell)$ операций, тогда как КОРА имеет трудоёмкость $O(|B|K\\ell)$.
    \item Параметр $T_1$ позволяет управлять жадностью алгоритма и находить компромисс между качеством конъюнкций и скоростью работы алгоритма.
    \item Благодаря простоте и эффективности алгоритм ТЭМП можно использовать в составе других алгоритмов как генератор конъюнкций, достаточно близких к оптимальным.
\end{itemize}

\subsection{Недостатки алгоритма ТЭМП}

\begin{itemize}
    \item Нет гарантии, что будут найдены самые лучшие конъюнкции, особенно при малых значениях параметра $T_1$.
    \item Алгоритм не стремится увеличивать различность конъюнкций, добиваясь равномерного покрытия объектов выборки. Стабилизация и редукция лишь отчасти компенсируют этот недостаток.
    \item Нет настройки коэффициентов $\\alpha_{tc}$; предполагается простое голосование.
\end{itemize}

\subsection{Задачи}

    \textbf{Задача 1.}

    \newline
    Рассмотрите граф $G = (V, E)$, где $V = \{A, B, C, D, E\}$ и $E = \{(A, B), (A, C), (B, D), (C, D), (D, E)\}$. Как алгоритм ТЭМП выполняет поиск в ширину для этого графа? Предложите, как можно выполнить поиск в глубину.

\textit{Решение:}
\begin{enumerate}
    \item Алгоритм ТЭМП начинает с вершины $A$, добавляя её соседей $B$ и $C$ в очередь. Затем из очереди выбирается вершина $B$, её сосед $D$ добавляется в очередь, и так далее.
    \item Для поиска в глубину из вершины $A$ алгоритм выбирает один путь (например, $A \to B \to D \to E$), пока не дойдёт до конца, затем возвращается и ищет другие пути.
\end{enumerate}

\textbf{Задача 2.}
\newline
Пусть у нас есть множество предикатов $B = \beta_1, \beta_2, \beta_3$ и максимальный ранг $K = 2$. Какое количество конъюнкций будет проверено алгоритмом ТЭМП, если для каждого шага выбирается $T_1 = 2$? Если мы увеличим $T_1$ до 3, как это повлияет на количество проверяемых конъюнкций?

\textit{Решение:}
\begin{itemize}
    \item При $T_1 = 2$ для ранга 1 будет 2 конъюнкции, для ранга 2 --- $2 \times 3 = 6$ конъюнкций. Всего будет проверено $2 + 6 = 8$ конъюнкций.
    \item Если $T_1 = 3$, то для ранга 1 будет 3 конъюнкции, для ранга 2 --- $3 \times 3 = 9$ конъюнкций. Всего будет проверено $3 + 9 = 12$ конъюнкций.
\end{itemize}

\textbf{Задача 3.}
\newline
Представьте, что вы используете алгоритм ТЭМП для обучения модели с большим количеством предикатов. Какой будет влияние на время вычислений, если вы увеличите $T_1$ с 1 до 5? Объясните это с точки зрения расширения пространства поиска.

\textit{Решение:}
\begin{itemize}
    \item При $T_1 = 1$ алгоритм будет проверять только по одному терму для каждой конъюнкции на каждом шаге, что будет довольно быстрым, но может не давать лучшие результаты.
    \item При увеличении $T_1$ до 5 пространство поиска расширяется, так как на каждом шаге будет добавляться больше термов к конъюнкциям, что увеличивает количество проверяемых комбинаций и тем самым замедляет процесс.
    \item В результате увеличение $T_1$ может существенно замедлить алгоритм, но при этом улучшится качество найденных решений.
\end{itemize}
